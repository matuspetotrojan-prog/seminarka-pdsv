\documentclass[12pt]{report}
\usepackage[slovak]{babel}
\usepackage[utf8]{inputenc}
\usepackage[T1]{fontenc}
\usepackage{csquotes}
\usepackage{geometry}
\usepackage{graphicx}
\usepackage{biblatex}
\usepackage{float}

\addbibresource{referencie.bib}



\geometry{a4paper, left=30mm, right=20mm, top=25mm, bottom=25mm}

\begin{document}
\begin{titlepage}
\centering


{\LARGE SLOVENSKÁ TECHNICKÁ UNIVERZITA}\\[6pt]
{\large FAKULTA ELEKTROTECHNIKY A INFORMATIKY}\\[2cm]


{\huge \textbf{Umelá inteligencia v dopravných systémoch}}\\[1.5cm]
{\Large Semestrálna práca}\\[1.5cm]

{Štúdijný program: Informatika}\\[0.3cm]
{Štúdijný odbor: Aplikovaná Informatika}\\[0.3cm]
{Predmet: Písanie dokumentov a správa verzií}\\[2cm]

\vfill
\noindent
\begin{minipage}[t]{0.5\textwidth}
\raggedright
\textbf{Matúš Trojan}
\end{minipage}%
\begin{minipage}[t]{0.5\textwidth}
\raggedleft
\textbf{\the\year}
\end{minipage}

\end{titlepage}


\tableofcontents
\listoffigures

\newpage

\section*{Abstrakt}

Táto práca skúma integráciu umelej inteligencie (AI) do moderných dopravných systémov. 
Cieľom práce je analýza technologických aspektov, praktických aplikácií a výziev spojených s implementáciou AI v doprave. 
Práca vychádza z rozsiahleho výskumného reportu Wisconsinského ministerstva dopravy a ďalších akademických zdrojov.

Práca je štruktúrovaná do piatich hlavných kapitol. 
V úvode sú definované kľúčové oblasti pôsobenia AI v doprave: manažment, bezpečnosť, operácie, autonómne vozidlá, digitálne dvojčatá a generatívna AI. 
Technologická časť detailne kľúčové architektúry a modely, vrátane strojového a hlbokého učenia, transformer architektúry a veľkých jazykových modelov. 
Kapitola o aplikáciách poskytuje prehľad implementácií vo všetkých druhoch dopravy s konkrétnymi príkladmi z praxe. 

Štvrtá kapitola venuje výzvam a etickým aspektom, analyzujúc problémy kvality dát, kybernetickej bezpečnosti, regulačného rámca, transparentnosti algoritmov a sociálno-ekonomických dopadov. 
Záver sumarizuje hlavné zistenia a navrhuje smerovanie budúceho vývoja.

Hlavným prínosom práce je komplexný prehľad interdisciplinárnej problematiky AI v doprave, ktorý spája technické aspekty s etickými a regulačnými úvahami. 
Práca dokladuje, že AI má potenciál revolučne zmeniť dopravné systémy, no jej úspešná integrácia si vyžaduje vyvážený prístup k technologickému pokroku a riadeniu súvisiacich rizík.

\noindent\textbf{Kľúčové slová:} umelá inteligencia, inteligentné dopravné systémy, autonómna doprava, generatívna AI, etika umelnej inteligencie, dopravné technológie

\newpage
\chapter{Úvod}
Umelá inteligencia (AI) sa v poslednom desaťročí stala jednou z najviac ovplyvňujúcich technológií 21. storočia, ktorá prináša revolúciu v mnohých priemyselných odvetviach vrátane dopravy.
,,Schopnosť AI analyzovať rozsiahle dátové súbory, rozpoznávať vzory, učiť sa z historických údajov a robiť prediktívne rozhodnutia otvára nové možnosti pre optimalizáciu dopravných systémov'' \cite{tech_report_2025}.
Integrácia AI do dopravy nie je len technologickou inováciou, ale nevyhnutnosťou pre riešenie dlhodobých výziev, ako sú dopravné preťaženia, bezpečnostné problémy a neefektívne využívanie infraštruktúry.

Dopravné systémy po celom svete sa potýkajú s rastúcou komplexitou, zvyšujúcim sa počtom vozidiel a meniacimi sa potrebami mobility.
,,Tradičné metódy manažmentu dopravných sietí, ako je manuálne monitorovanie premávky a údržba infraštruktúry, už nestačia na efektívne zvládanie týchto požiadaviek'' \cite{tech_report_2025}.
Práve v tomto kontexte ponúka AI nové možnosti na optimalizáciu dopravy prostredníctvom pokročilých algoritmov strojového učenia, spracovania obrazu a analýzy veľkých dát.
\section{Historický kontext a vývoj}
Využitie AI v doprave má svoje korene v jednoduchších algoritmoch na riadenie dopravy v 70. a 80. rokoch minulého storočia. 
S rozvojom výpočtového výkonu a dostupnosťou dát v 21. storočí však došlo k významnému zrýchleniu implementácie AI riešení. 
,,Ministerstvo dopravy USA (USDOT) vo spolupráci s Federálnou správou diaľnic (FHWA), Federálnou železničnou správou (FRA) a Federálnou leteckou správou (FAA) začalo aktívne preskúmať potenciál AI v dopravnom ekosystéme'' \cite{tech_report_2025}.
Tieto snahy svedčia o systémovom prístupu k integrácii AI do dopravnej infraštruktúry.

\section{Kľúčové oblasti aplikácie AI v doprave}

V súčasnosti sa výskum a implementácia AI v doprave sústredí na šesť hlavných domén, ktoré sú identifikované ako kľúčové pre transformáciu dopravných systémov:

\subsection{Manažment dopravných aktív}
AI významne zlepšuje efektivitu monitorovania a údržby dopravnej infraštruktúry. 
,,Aplikácie zahŕňajú automatizované vyhodnocovanie stavu vozoviek a mostov, prediktívnu údržbu a optimalizáciu zdrojov'' \cite{tech_report_2025}. 
,,Štúdie ukazujú, že AI techniky ako počítačové videnie a hlboké učenie môžu detekovať defekty ako trhliny alebo koróziu s vysokou presnosťou, čím sa znižuje potreba manuálnych inšpekcií a zvyšuje sa bezpečnosť'' \cite{tech_report_2025}.


\subsection{Dopravná bezpečnosť}
AI aplikácie v dopravnej bezpečnosti zahŕňajú systémy na predikciu nehôd, monitorovanie zraniteľných účastníkov premávky a hodnotenie nebezpečenstiev na cestách. 
,,Techniky počítačového videnia a strojového učenia umožňujú spracovanie dát v reálnom čase a neustále zlepšovanie bezpečnostných systémov'' \cite{tech_report_2025}. 
,,Hawaii Department of Transportation (HDOT) sa napríklad zapojil do Intersection Safety Challenge implementáciou systémov na báze AI pre detekciu zraniteľných účastníkov premávky a vylepšenie systémov včasného varovania ''\cite{tech_report_2025}.

\subsection{Dopravné operácie}
,,AI optimalizuje dopravné operácie prostredníctvom aplikácií ako ramp metering, predikcia toku premávky, optimalizácia časovania semaforov a variabilné obmedzenia rýchlosti'' \cite{tech_report_2025}. 
,,Florida Department of Transportation (FDOT) implementoval nástroje na riadenie premávky využívajúce AI na optimalizáciu časovania semaforov pozdĺž mestských tepien s cieľom znížiť kongescie v špičke'' \cite{tech_report_2025}.

\subsection{Autonómne vozidlá}
Autonómne vozidlá predstavujú jednu z najtransformujúcejších aplikácií AI v doprave.
,,Illinois Department of Transportation (IDOT) spustil iniciatívu "Autonomous Illinois", ktorá vytvára pilotné zóny pre testovanie autonómnych vozidiel v reálnom prostredí prostredníctvom spolupráce verejného a súkromného sektora a akademickej obce'' \cite{tech_report_2025}.

\subsection{Digitálne dvojčatá}
,,Digitálne dvojčatá sú virtuálnymi replikami fyzických aktív alebo systémov, ktoré využívajú údaje v reálnom čase na simuláciu, analýzu a optimalizáciu operácií'' \cite{tech_report_2025}. 
Ako je znázornené na obrázku \ref{fig:digital_twin}, táto technológia umožňuje vytvorenie komplexného prepojenia medzi fyzickou infraštruktúrou a jej digitálnym modelom.
\begin{figure}[H]
    \centering
    \includegraphics[width=0.95\textwidth]{digitalnedvojca.png}
    \caption{Schéma digitálneho dvojčaťa mostu. Zobrazený je proces prepojenia fyzického mostu s jeho virtuálnym modelom, vrátane monitorovania záťaže, identifikácie poškodenia, vyhodnocovania výkonu a systémov včasného varovania.(Zdroj:\cite{Kang2021}) }
    \label{fig:digital_twin}
\end{figure}

\subsection{Generatívna AI}
Generatívna AI nachádza uplatnenie v doprave najmä v autonómnej jazde, predikcii premávky a tvorbe realistických simulačných scenárov. 
,,California Department of Transportation (Caltrans) testuje využitie generatívnej AI na sumarizáciu dokumentov a generovanie interných správ, čím zefektívňuje pracovné postupy a zlepšuje správu znalostí'' \cite{tech_report_2025}.

\chapter{Technologické aspekty AI v doprave}

Umelá inteligencia predstavuje komplexný ekosystém technológií, ktorých úspešná integrácia do dopravy vyžaduje pochopenie základných komponentov.
Táto kapitola poskytuje prehľad kľúčových technológií poháňajúcich moderné inteligentné dopravné systémy (ITS), s dôrazom na generatívnu AI.

\section{Kľúčové technologické modely}

\subsection{Symbolická vs. dátovo-riadená AI}
Historický vývoj AI sa delí na dva modely.
,,Symbolická AI, založená na manuálne vytvorených pravidlách, bola obmedzená v komplexných prostrediach ako doprava'' \cite{EPRS_2019}.
,,Dátovo-riadená AI, využívajúca strojové učenie a analýzu veľkých dát, umožňuje objavovať vzory a robiť predikcie, čo z nej urobilo jadro moderných ITS'' \cite{EPRS_2019}.

\subsection{Techniky strojového a hlbokého učenia}
Hlboké učenie, ako podmnožina strojového učenia, umožňuje spracovanie komplexných dát.
,,Konvolučné neurónové siete (CNN) sú nevyhnutné pre spracovanie obrazu z kamier, zatiaľ čo rekurentné siete (RNN) sú vhodné na analýzu časových radov pre predikciu dopravnej záťaže'' \cite{Rong2025}.

\section{Generatívna AI a veľké jazykové modely}
Generatívna AI prináša kvalitatívny skok tým, že dokáže generovať nový obsah namiesto len analýzy existujúcich dát.

\subsection{Architektúra Transformers a LLM}
,,Model Transformer (2017) revolucionalizoval spracovanie sekvenčných dát pomocou attention mechanizmu'' \cite{Vaswani2017}.
,,Stal sa základom pre veľké jazykové modely (LLM) ako GPT-3, ktoré dokážu komplexne uvažovať a sú použiteľné na interpretáciu dopravných scenárov a analýzu nehôd'' \cite{Brown2020, Rong2025, Zheng2023}.

\subsection{GAN a difúzne modely}
,,Generatívne adversariálne siete (GAN) vytvárajú realistické obrazy prostredníctvom súboja generátora a diskriminátora'' \cite{Goodfellow2014}.
,,Difúzne modely generujú obsah postupným odstraňovaním šumu, čo je ideálne pre tvorbu simulácií pre testovanie autonómnych vozidiel'' \cite{Ho2020, Rong2025}.

\section{Integrácia AI do subsystémov ITS}
,,ITS pozostáva zo štyroch subsystémov: cestného, vozidlového, cestovateľského a riadiaceho'' \cite{Meneguette2018, Alams2016}.
,,AIGC sa do nich integruje cez tri oblasti: dialóg a uvažovanie, predikcia a rozhodovanie, a multimodálna generácia'' \cite{Rong2025}.

\subsection{Dialóg a uvažovanie}
,,LLM vylepšujú komunikáciu v ITS. Inteligentní virtuálni asistenti vo vozidlách poskytujú personalizované navigačné pokyny'' \cite{Rong2025}.
,,Zatiaľ čo automatická analýza nehôd urýchľuje určovanie príčin a zodpovednosti'' \cite{Chen2024, Zhou2024GPT4V}.

\subsection{Predikcia a rozhodovanie}
,,AI modely optimalizujú časovanie semaforov a dynamické smerovanie premávky'' \cite{Pompigna2022}.
,,Pokročilé modely ako ST-LLM predpovedajú dopyt po prepravných prostriedkoch s vysokou presnosťou'' \cite{Liu2024STLLM}.

\subsection{Multimodálna generácia}
,,AIGC generuje fotorealistické scenáre pre testovanie autonómnych vozidiel, čím zvyšuje bezpečnosť a efektivitu testovania'' \cite{Xu2024GenerativeAI, Wang2024MDT}.
,,Kombinácia s digitálnymi dvojčatami umožňuje simulovať vplyv dopravných opatrení'' \cite{Wang2024MDT}.

\section{Technologická infraštruktúra}

\subsection{Senzory}
,,Dopravné systémy sú nasýtené senzormi (kamery, radary, GPS), ktoré tvoria internet vecí a produkujú dáta pre AI analýzy'' \cite{EPRS_2019}. 
Typickú konfiguráciu takýchto senzorov v autonómnom vozidle ilustruje obrázok \ref{fig:sensors}.
\begin{figure}[H]
    \centering
    \includegraphics[width=0.8\textwidth]{autonomouscarsensors.png}
    \caption{Ilustrácia senzorového systému autonómneho vozidla. Zobrazené sú typické senzory: dlhodosahový radar vpredu, dlhodosahový LiDAR vpredu, predná kamera, okolité kamery, 360° otočný LiDAR, bočné radary a LiDARy so širokým zorným poľom, ako aj riadiaca jednotka ADAS/AD. (Zdroj: IDTechEx Research)}
    \label{fig:sensors}
\end{figure}

\noindent\textbf{Vysvetlenie jednotlivých komponentov:}

\begin{itemize}
    \item \textbf{Radar:} Vhodný na meranie rýchlosti a vzdialenosti, funguje za akéhokoľvek počasia
    \item \textbf{LiDAR:} Poskytuje presné 3D mapovanie prostredia, citlivý na poveternostné podmienky
    \item \textbf{Kamery:} Rozpoznávajú farby, texty a detaily, vyžadujú dobré osvetlenie
    \item \textbf{ADAS/AD jednotka:} Spája dáta zo všetkých senzorov a rozhoduje o ďalšej trase
\end{itemize}
\subsection{Edge a cloud computing}
,,Edge computing umožňuje okamžitú analýzu priamo v zariadení, zatiaľ čo cloud computing poskytuje výkon pre trénovanie veľkých AI modelov'' \cite{Rong2025}.

\subsection{Komunikačné technológie}
,,Komunikácia vozidlo so všetkým (V2X) a 5G siete sú nevyhnutné pre prenos dát v reálnom čase pre kooperatívne autonómne vozidlá'' \cite{EPRS_2019}.

\chapter{Aplikácie AI v dopravných systémoch}

Umelá inteligencia nachádza praktické využitie vo všetkých druhoch dopravy, pričom každý systém má svoje špecifické výzvy.
Táto kapitola analyzuje kľúčové aplikácie AI v cestnej, železničnej, leteckej a námornej doprave, s dôrazom na reálne implementácie a ich vplyv.

\section{Cestná doprava}

Cestná doprava je kde sa AI aplikuje najviditeľnejšie.
,,Cestná doprava je jedným z odvetví, kde sa AI najúspešnejšie uplatňuje, otvárajúc úplne novú úroveň spolupráce medzi rôznymi účastníkmi premávky'' \cite{EPRS_2019}.

\subsection{Autonómne vozidlá}
Vývoj autonómnych vozidiel je jednou z najprelomovejších aplikácií AI.
,,Autonómne vozidlá sú založené na rôznych senzorch, ako sú GPS, kamery a radary, v kombinácii s aktuátormi, riadiacimi jednotkami a softvérom'' \cite{EPRS_2019}.
Typickú konfiguráciu takýchto senzorov ilustruje obrázok \ref{fig:sensors} v kapitole 2.
Niektoré technológie preberajú iba určité funkcie vodiča (ako parkovanie), zatiaľ čo iné sú určené na úplné nahradenie vodiča \cite{EPRS_2019}.

\subsection{Riadenie premávky}
,,AI technológie sa tiež uplatňujú pri riadení cestnej premávky, pomáhajúc analyzovať vzory premávky, jej objem a ďalšie faktory'' \cite{EPRS_2019}.
Tieto systémy poskytujú vodičom informácie o najrýchlejšej trase, čím zmierňujú dopravné zápchy.
,,AI technológie tiež pomáhajú udržiavať plynulosť premávky prostredníctvom dopravných signálov a semaforov, ktoré sa v reálnom čase prispôsobujú aktuálnym dopravným podmienkam'' \cite{EPRS_2019}.

\section{Železničná doprava}

Železničná doprava využíva AI predovšetkým na automatizáciu, údržbu a bezpečnosť.
,,AI môže zlepšiť výrobu, prevádzku a údržbu pre železničných operátorov a manažérov infraštruktúry'' \cite{EPRS_2019}.

\subsection{Automatizácia prevádzky vlakov (ATO)}
,,Jedným z najvýrečnejších príkladov využitia AI v železničnej technike je jej prínos k automatizácii prevádzky vlakov'' \cite{EPRS_2019}.
Medzinárodná elektrotechnická komisia stanovila štyri štandardné stupne automatizácie vlakov, pričom tretí stupeň zodpovedá bezvodičovej prevádzke a štvrtý autonómnej a neobsluhovanej prevádzke vlakov \cite{EPRS_2019}.

\subsection{Európsky systém riadenia železničnej premávky (ERTMS)}
,,V EU je prvým kľúčovým krokom k zavedeniu ATO a riešení AI v železničnej doprave nasadenie európskeho systému riadenia železničnej premávky'' \cite{EPRS_2019}.
Okrem zabezpečenia technickej kompatibility medzi národnými železničnými systémami môže ERTMS v kombinácii s ATO znížiť náklady železničných operátorov a spotrebu energie, ako aj zvýšiť rýchlosť, dochvíľnosť, bezpečnosť a kapacitu tratí \cite{EPRS_2019}.

\subsection{Prediktívna údržba}
,,Pre železničných operátorov a manažérov infraštruktúry je schopnosť predvídať možné poruchy predtým, ako k nim dôjde, veľmi cenná'' \cite{EPRS_2019}.
,,Dnes môže AI využiť silu údajov poskytovaných senzormi umiestnenými na kritických komponentoch vlakov alebo infraštruktúry na extrahovanie informácií v správnom čase a odporúčanie opatrení na údržbu'' \cite{EPRS_2019}.

\section{Letecká doprava}

Letecký priemysel využíva AI  už desaťročia, no vstupujeme do novej éry s pokročilými možnosťami.
,,Použitie AI v prevádzke leteckej dopravy je ešte v počiatkoch'' \cite{EPRS_2019}.

\subsection{Riadenie letovej prevádzky (ATM)}
,,Pokroky v automatizácii a výpočtovej sile, využívajúce technológie spojené s strojovým učením a modelmi analýzy údajov, sa používajú na zlepšenie riadenia rastúcich objemov leteckej dopravy'' \cite{EPRS_2019}.
Vývoj systémov bezpilotných leteckých systémov a systémov riadenia premávky bezpilotných systémov pomocou vylepšených výpočtových schopností vytvorí nové príležitosti na zlepšenie existujúcich systémov riadenia premávky, noriem odstupu a návrhu priestoru \cite{EPRS_2019}.

\subsection{Letiskové operácie a bezpečnosť}
,,AI môže uľahčiť prechod k bezproblémovej bezpečnosti letísk, pretože je schopná spracovať veľké množstvo údajov, a to historických aj v reálnom čase, a detegovať anomálie'' \cite{EPRS_2019}.
Použitie AI a veľkých dát sa pre letiská stáva čoraz dôležitejšie a používa sa na lepšiu analýzu dopytu na trhu, zlepšenie bezpečnostnej kontroly a prispôsobenie skúsenosti cestujúcich \cite{EPRS_2019}.

\subsection{Výskumné projekty EÚ}
,,Spoločný podnik SESAR podporil množstvo výskumných projektov týkajúcich sa AI a riadenia leteckej dopravy'' \cite{EPRS_2019}.
Projekt INTUIT skúmal potenciál strojového učenia a vizuálnej analytiky, zatiaľ čo projekt COPTRA sa zameral na predpovedanie trajektórií bližšie ku štartu alebo počas letu \cite{EPRS_2019}.

\section{Námorná doprava, plavba a prístavy}

Vodná doprava prechádza významnou digitálnou transformáciou s AI.
,,V priebehu posledných dvadsiatich rokov prešli námorná a vnútrozemská vodná doprava dôležitým vývojom'' \cite{EPRS_2019}.

\subsection{Autonómna plavba}
,,Hoci medzera medzi súčasnou situáciou a plne autonómnymi loďami je veľká, výskum a pilotné projekty autonómnej plavby pokračujú'' \cite{EPRS_2019}.
Projekt MUNIN financovaný EÚ vyvinul a otestoval koncept autonómnej obchodnej lode, ktorou primárne riadia automatizované palubné rozhodovacie systémy, ale kontroluje ju vzdialený operátor na pevnine \cite{EPRS_2019}.
,,Existuje tiež potenciál pre použitie autonómnych lodí v pobrežnej plavbe a na vnútrozemských vodných cestách'' \cite{EPRS_2019}.

\subsection{Inteligentné prístavy}
,,Použitie pokročilých digitálnych technológií v celom prostredí prístavu je známe ako koncept ,,inteligentného'' alebo ,,smart'' prístavu'' \cite{EPRS_2019}.
,,Medzi prístavy považované za najpokročilejšie v premenení na inteligentný prístav patria Singapore, Rotterdam, Tianjin a Dubaj'' \cite{EPRS_2019}.

\subsection{Optimalizácia prístavných operácií}
,,AI je len jednou z niekoľkých kľúčových technológií používaných v inteligentnom prístave'' \cite{EPRS_2019}.
,,V prístavných prevádzkových systémoch sa používa napríklad na plánovanie vybavenia prístavu (na optimalizáciu použitia žeriavov a vozidiel) a plánovanie dostupnosti kotvišť'' \cite{EPRS_2019}.
,,Prístav Rotterdam aplikuje AI na údaje na určenie odhadovaného času príchodu a odchodu lode, čo pomohlo znížiť čakaciu dobu pre plavidlá v prístave o 20\%'' \cite{EPRS_2019}.

\subsection{Výzvy v námornej doprave}
,,Pre akékoľvek použitie AI sú zber, kvalita, konzistentnosť a objem dostupných údajov prvoradé'' \cite{EPRS_2019}.
,,Niektoré problémy s kvalitou a množstvom údajov vznikajú pri údajoch o výkone a navigácii lodí zhromažďovaných senzormi a systémami na získavanie údajov'' \cite{EPRS_2019}.
,,Údaje môžu byť chybné kvôli poruchám senzorov alebo ľudskej intervencii'' \cite{EPRS_2019}.

\chapter{Výzvy a etické aspekty implementácie AI v doprave}

Integrácia umelej inteligencie do dopravy prináša okrem príležitostí aj zložité výzvy. 
Táto kapitola analyzuje problémy technického a etického charakteru.

\section{Technické a bezpečnostné výzvy}

Kvalita dát je základným predpokladom úspechu AI systémov. 
,,Pre akékoľvek použitie AI sú zber, kvalita, konzistentnosť a objem dostupných údajov prvoradé'' \cite{EPRS_2019}. 
V praxi sa často stretávame s neúplnými alebo nekonzistentnými dátami z rôznych zdrojov.

Kybernetická bezpečnosť predstavuje kritickú oblasť. 
,,AI aplikácie v doprave vyvolávajú obavy týkajúce sa kybernetickej bezpečnosti, najmä v kontexte autonómnych vozidiel a systémov riadenia dopravy'' \cite{EPRS_2019}. 
Úspešný útok môže mať katastrofálne následky.

Výskum generatívnej AI identifikuje ďalšie riziká. 
,,AIGC je náchylné k problému ,,halucinácií'', kde generovaný obsah nemusí zodpovedať skutočnosti'' \cite{Rong2025}. 
V dopravnom prostredí môže nepresná detekcia alebo chybné rozhodovanie viesť k nehodám.

\section{Legislatívne a regulačné výzvy}

Určenie zodpovednosti pri nehodách autonómnych systémov je komplexnou otázkou. 
,,V prípade nehody je ďalšou výzvou, ktorú je potrebné riešiť, zodpovednosť'' \cite{EPRS_2019}. 
Súčasné pravidlá predpokladajú ľudského vodiča, čo sa mení s autonómnymi systémami.

EÚ aktívne pracuje na prispôsobení regulačného rámca. 
,,V decembri 2018 Komisia zverejnila návrh etických smerníc AI a posudzuje, či sú národné a európske bezpečnostné a zodpovednostné rámce vhodné'' \cite{EPRS_2019}. 
Harmónia medzinárodných predpisov je kľúčová pre globálnu implementáciu.

\section{Etické problémy}

Trolley problém v autonómnych vozidlách rozdeľuje odbornú verejnosť. ,,Keď čelia situáciám život verzus život, otázka, ako by sa mal rozhodovať algoritmus AI v plne automatizovanom vozidle, rozdeľuje názory'' \cite{EPRS_2019}. 
Etické nastavenie algoritmov vyžaduje širokú diskusiu.

Transparentnosť rozhodovania je základom dôvery. ,,Princíp transparentnosti by mal byť dodržaný tým, že by malo byť vždy možné poskytnúť zdôvodnenie akéhokoľvek rozhodnutia prijatého s pomocou AI'' \cite{EPRS_2019}. 
Problém ,,čiernej skrinky'' komplikuje pochopenie rozhodovacích procesov.

Generatívna AI prináša nové etické výzvy. 
,,Aplikácia AIGC vyvoláva právne a etické obavy... modely nemusia plne dodržiavať existujúce predpisy a potenciálne zraniteľnosti alebo nepresné odpovede môžu viesť k nehodám'' \cite{Rong2025}. 
Ochrana duševného vlastníctva a ochrana súkromia sú ďalšie problémy.

\section{Sociálno-ekonomické dopady}

AI mení štruktúru pracovného trhu v dopravnom sektore. 
,,Predpokladá sa, že AI prispieje k vytvoreniu nových pracovných miest, zániku iných a zmene väčšiny'' \cite{EPRS_2019}. 
Zatiaľ čo niektoré pozície môžu zaniknúť, vznikajú nové role vyžadujúce špecifické zručnosti.

Prijatie technológií verejnosťou je kľúčové pre ich úspech. 
,,Nedávny prieskum Eurobarometer o autonómnych systémoch ukázal, že respondenti sú pohodlnejší s autonómnymi vozidlami prepravujúcimi tovar ako s cestovaním v takom vozidle sami'' \cite{EPRS_2019}.
Dôvera verejnosti sa buduje pomaly a vyžaduje transparentnosť.

Generatívna AI prináša výzvy v oblasti ľudsko-strojovej spolupráce. 
,,S rozvojom AIGC sú tradičné pracovné trhy výrazne narušené... úlohy ľudí sa už neobmedzujú na vykonávanie opakujúcich sa úloh, ale teraz vyžadujú schopnosť efektívne spolupracovať s novými technológiami'' \cite{Rong2025}. 
Táto transformácia si vyžaduje aktívne prispôsobenie vzdelávacích systémov.

\section{Budúce smerovanie a riešenia}

Medzinárodná spolupráca a štandardizácia sú nevyhnutné. 
,,Európsky parlament vyzval na vytvorenie európskej agentúry pre robotiku a AI a požiadal Komisiu, aby predložila návrh legislatívneho nástroja o právnych otázkach súvisiacich s vývojom a používaním robotiky a AI'' \cite{EPRS_2019}. 
Koordinovaný prístup znižuje fragmentáciu.

Vzdelávanie a rekvalifikácia sú kľúčové pre prispôsobenie sa zmene. 
,,Títo vodiči budú potrebovať rekvalifikáciu, aby si našli inú prácu'' \cite{EPRS_2019}.
Transformácia pracovného trhu si vyžaduje aktívne opatrenia v oblasti vzdelávania a odbornej prípravy.

Budúci výskum by sa mal sústrediť na komplexné riešenia. 
,,Existujúci výskum sa väčšinou zameriava na špecifické subsystémy a technológie... budúci výskum by sa mal zamerať na vývoj univerzálneho veľkého modelu, ktorý integruje tieto tri AIGC technológie a koordinuje štyri subsystémy ITS'' \cite{Rong2025}. 

\chapter*{Záver}
\addcontentsline{toc}{chapter}{Záver}

Integrácia AI do dopravy predstavuje proces s možnosťami pre efektivitu, bezpečnosť a udržateľnosť, ale aj s výraznými výzvami. Táto práca poskytla prehľad stavu, technológií a aplikácií AI vo všetkých dopravných odvetviach.

Hlavný prínos AI spočíva v schopnosti spracovať veľké objemy dát v reálnom čase, čo umožňuje:
\begin{itemize}
    \item \textbf{Optimalizáciu prevádzky} adaptívnym riadením premávky a inteligentným plánovaním
    \item \textbf{Zvýšenie bezpečnosti} systémami na predikciu nehôd a podporu autonómnych systémov
    \item \textbf{Zlepšenie komfortu} personalizovanými službami a znížením zápch
\end{itemize}

Napriek potenciálu existujú kritické bariéry:
\begin{enumerate}
    \item \textbf{Technologické:} Spoľahlivosť systémov, kvalita dát a kybernetická bezpečnosť
    \item \textbf{Regulačné:} Chýbajúci legislatívny rámec pre autonómne systémy a problematika zodpovednosti
    \item \textbf{Etické:} Ochrana súkromia, ,,problém čiernej skrinky'' a dôvera verejnosti
\end{enumerate}

Budúci vývoj by sa mal zamerať na adaptívne AI modely, medzinárodnú spoluprácu pri tvorbe noriem, vzdelávanie pracovnej sily a výskum pokročilých technológií ako LLM.

AI už nie je víziou, ale reálnym nástrojom, ktorý mení dopravu. 
Úspešná integrácia si vyžaduje komplexný prístup zohľadňujúci technické, etické a sociálne aspekty.
\newpage
\printbibliography[title={Zoznam referencií}]
\section*{Upozornenie o používaní AI}

V tejto práci boli vygenerované úvod, abstrakt a záverečná časť s pomocou veľkého jazykového modelu DeepSeek.
Obsahová štruktúra, citácie a analytická časť práce (kapitoly 2-4) sú výsledkom vlastnej práce autora a analýzy zdrojov.
\end{document}